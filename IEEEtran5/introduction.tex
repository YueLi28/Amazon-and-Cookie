\section{Introduction}
Entering the era of big data, Internet has been instantly extended to be data-rich. User data can be exploited to generate large benefit in many fields. Electronic commerce is one of the most data-driven market. There have been many works studying how users behave or user reactions in E-commercials \cite{tsai2011effect, ghose2011estimating, ivanova2013does}. Studying user behaviors in e-commercials help understanding the market and thus generate better economic outcomes. For example, user data can be monetized by feeding targeted advertising or performing price discrimination~\cite{mikians2012detecting}. Although there are various kinds of user data on Amazon, user shopping history and preference is one of the most sensitive and valuable type of data. However, shopping history is considered private information that most users do not wish to publicize. To cope with user expectations, most e-commerce companies keep user shopping history private. Therefore, a major challenge of studying shopping preference is to collect data from users. Nonetheless, users may not always hide their purchase intentions. Wishlist, a list-type data in Amazon, is made publicly available by default. Users add items in their wishlists to record desired products for their own reference or for gift givers. As users use wishlists to record desired products, wishlists largely reflect the shopping preference of users -- If a user adds a item in his/her wishlist, the user is likely to buy the item or make other people buy the item. Besides, wishlists are also indicators of shopping history since the items will not be removed after the item is bought unless manually deleted.

As the world's leading e-commerce company, Amazon is a particularly important source for understanding user shopping behaviors. In this paper, we analyze wishlists in Amazon to study user shopping patterns as well as the privacy implication of wishlists. By doing so we shed light on general user preference on e-commerce and help companies to refine both their marketing strategies and privacy policy.

We first collected complete profile and wishlists information of over 30,000 users and approximately 2 million items in Amazon by web scraping. Based on the data, we conduct measurement study on user behaviors. Our study concentrates on user shopping preference, which are organized in 3 dimensions — 1)Product Categories 2)Product prices 3) Timing. Specifically, we compare the user preference in different gender and regions. We also investigate shopping pattern in different time through a year. Surprisingly we found that there is little shopping increase in holidays or weekends and even some holidays are shopping repelling. We believe our findings provide insight that is able to help sellers to revise their marketing strategies, as well as to help advertisers to make more accurate targeted advertising. 

Beside user behavior analysis, we study user privacy information exposure in Amazon. Network users are prone to expose their private information in public websites inadvertently~\cite{frankowski2006you, friedland2010cybercasing}, making themselves face the threat of information leakage. Privacy protection is very important for websites not only because it involves legal issues but also that it correlates to user purchasing intentions~\cite{brown2004investigating, tsai2011effect}. To study to what extent does the information in wishlists threat user privacy, we investigate both the items in wishlists and the user input list-descriptions to identify user personal information. We first illustrate user personal information exposure by analyzing their list-descriptions, showing that users are mentioning sensitive personal information such as profession, education background, relatives' information, etc in their list-descriptions. 

Furthermore, public information may be exploited to infer user personal information. Such information leakage has been proved in many works~\cite{narayanan2009anonymizing, wondracek2010practical, chaabane2012you, hecht2011tweets, goga2013exploiting}. In our study, we also try to identify user personal information leakage through analyzing publicly available data. Specifically, we use Support Vector Machine (SVM) to predict user gender based solely on items in their wishlists. The result indicates that user gender information can be identified with fairly high accuracy.  
