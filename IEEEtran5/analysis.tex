\section{Data Analysis}

Now we aim to dissect our dataset to reveal how users use their wishlists. Particularly we are interested in the products user prefer to buy, as well as the time that is appealing to shoppers. We compare user preferences in 3 different regions to show market trend. Besides, we also quantify the increase in different time of a year. Specifically we show different shopping behaviors in various holidays.

\subsection{Basic Statistics}
First we conduct fundamental measurements on the dataset to help build basic understanding on how wishlists are used and to what extend the users expose their personal information in Amazon. 

For all 967,603 users collected in the first step of data collection, we record totally 2,121,173 wishlists and 5,700,000 items (in which 2,248,142 items are unique). We show distribution metrics in Table~\ref{tb:stat2}. As we can see from the table, every user has 2.56 wishlists in average with standard deviation of 4.97. The average number and standard deviation of items in a wishlist is 74.2 and 187.3 accordingly. And the average and standard deviation of product price is \$35.06 and \$172.0. Note that although we list the maximum number of wishlists a user has (Similarly, items in wishlists and item price) in our dataset, we did not probe the limit set by Amazon. All three distributions have very high skewness~($\gamma_1$) and kurtosis~($\kappa$), which means the distributions are very skewed and heavily tailed. For clearer presentation, we show the distribution of wishlist in Figure~\ref{avglist}, distribution of items in Figure~\ref{avgitem}, and distribution of price in Figure~\ref{avgprice} using log-scaled y axis.

\begin{table}[t]
\centering
\caption{Distributions}
\label{tb:stat2}
\begin{adjustbox}{max width=.5\textwidth}
\begin{tabular}{llllll}
Distribution & Mean & Max & SD & $\gamma_1$ & $\kappa$ \\ \hline
Wishlist in Profile & 2.56 & 326 & 4.97 & 19.7 & 854.15 \\
Items in Wishlist & 74.2 & 7,350 & 187.3 & 8.25 & 111.11 \\
Item Price & 35.06 & 105,065 & 172.0 & 299.4 & 144,743.2 \\
\end{tabular}
\end{adjustbox}
\end{table}

\begin{figure*}[t]
\minipage{0.32\textwidth}
  \includegraphics[width=\linewidth]{avglist.png}
  \caption{Number of lists the users have}\label{avglist}
\endminipage\hfill
\minipage{0.32\textwidth}
  \includegraphics[width=\linewidth]{avgitem.png}
  \caption{Number of items the lists have}\label{avgitem}
\endminipage\hfill
\minipage{0.32\textwidth}%
  \includegraphics[width=\linewidth]{avgprice.png}
  \caption{Item Price}\label{avgprice}
\endminipage
\end{figure*}


\subsection{User preference}
A core question on user preference is what the users would like to buy and how much they are willing to pay for certain products. To categorize the products, we directly use item type that are in the product pages (See Fig~\ref{item}. From all product pages that are visited we found 50 types in total. Note that Amazon was reported to do price discrimination on E-books~\cite{mikians2012detecting}, which indicates a same E-book may be priced differently in various locations. However, these locations are in scope of countries. In our study, we focus on the U.S. Besides, we believe our results are also meaningful in a global perspective since only E-books are price discriminated. Most of our conclusions still stand.

We show national user preference in Table~\ref{tb:overall}. With over 40\% of items in wishlist being books, we can see books are in domination. Beside normal paperback books, E-books in Kindle are also being very popular. We believe the cheap price of E-book is a main reason for its prosperity (Books are 136.7\% more expensive than E-books in average). Other than books, entertainment products play second important roles. Movies \& TV, CDs \& Vinyl, Toys \& Games, and Video games rank 2, 4, 5, 6 accordingly, which indicate that users are normally pursuing leisure products on Amazon. The following popular products are fashions, home related, and sport items. We also found that Computers, All Electronics, and Camera are only 3 types of products users are paying more than \$100 in average, which shows that users are paying substantially more in electronic devices than other products.


\begin{table}[t]
\centering
\caption{Overall User Preference}
\label{tb:overall}
\begin{adjustbox}{max width=.5\textwidth}
\begin{tabular}{lllll}
Rank & Item Type          & Number of Items & Percentage of Items & Average Price(\$) \\ \hline
1 & Books & 1829657 & 41.55\% & \$22.25 \\
2 & Movies \& TV & 533504 & 12.12\% & \$25.26 \\
3 & Buy a Kindle & 391967 & 8.90\% & \$9.40 \\
4 & CDs \& Vinyl & 337188 & 7.66\% & \$17.36 \\
5 & Toys \& Games & 231522 & 5.26\% & \$40.42 \\
6 & Video Games & 103951 & 2.36\% & \$46.87 \\
7 & Amazon Fashion & 102012 & 2.32\% & \$57.77 \\
8 & Kitchen \& Dining & 99647 & 2.26\% & \$46.00 \\
9 & Sports \& Outdoors & 95522 & 2.17\% & \$59.16 \\
10 & Home \& Kitchen & 85088 & 1.93\% & \$58.50 \\
11 & Home Improvement & 72194 & 1.64\% & \$62.49 \\
12 & All Electronics & 58146 & 1.32\% & \$121.55 \\
13 & Health \& Personal Care & 47912 & 1.09\% & \$34.54 \\
14 & Digital Music & 38375 & 0.87\% & \$8.94 \\
15 & Computers & 38066 & 0.86\% & \$123.31 \\
16 & All Beauty & 37946 & 0.86\% & \$23.45 \\
17 & Camera \& Photo & 37725 & 0.86\% & \$192.99 \\
18 & Arts, Crafts \& Sewing & 31029 & 0.70\% & \$24.91 \\
19 & Patio, Lawn \& Garden & 30747 & 0.70\% & \$66.83 \\
20 & Grocery \& Gourmet Food & 27616 & 0.63\% & \$21.67 \\
\end{tabular}
\end{adjustbox}
\end{table}

After analyzing the general shopping preference of Amazon users, it is natural to study the preferences of people from different demographical and geographical groups. For demographical factor, we compare the shopping preference of males and females. For geographical factor, we compare customers from 3 different geo-location \footnote{East coast includes Maine, New Hampshire, Massachusetts, Rhode Island, Connecticut, New York, New Jersey, Delaware, Maryland, Virginia, North Carolina, South Carolina and Georgia. West coast includes California, Oregon and Washington. Middle of the US includes the rest states}, which are -- east coast, west coast, and middle of the US. 

\subsubsection{Different Gender}
\label{differentgender}
To start with, we show the top 10 popular products of male users in Table~\ref{tb:male} and female users in Table~\ref{tb:female}. In addition, the average product price for male and female is \$36.95 and \$26.53 correspondingly. Clearly we can find that male and female customers have different shopping preference. First of all, products in males' wishlists have 39.3\% higher price than those in females' wishlists averagely. Furthermore, males are willing to pay higher price in all categories in the two tables. The gap between the product price is even larger than the income gap between males and females, which are reported that males earn 21.1\% more than females in 2014~\footnote{http://www.catalyst.org/knowledge/womens-earnings-and-income}. Therefore we concluded that males are prone to spend more online. Beside the price difference, different genders also prefer different categories of products. While ``Books" and ``Buy a Kindle" account around 50\% of the products for both genders, male customers are more likely to buy a paperback book instead of E-books than female customers. Males prefer sports and electronics while females prefer fashions and beauty-related items. Interestingly, females like ``Arts, Crafts \& Sewing" much more than males as such items count only 0.3\% of all products and ranks 26 in all categories for males. In general, 13 out of the most popular 24 categories (both male and female have over 5000 products in these categories) have over 100\% difference, which means that the percentage of certain product of a gender is at least double as the other gender. 

\begin{table}[t]
\centering
\caption{Male User Preference}
\label{tb:male}
\begin{adjustbox}{max width=.5\textwidth}
\begin{tabular}{lllll}
Rank & Item Type          & Number of Items & Percentage of Items & Average Price(\$) \\ \hline
1 & Books & 1306784 & 43.56\% & \$24.16 \\
2 & Movies \& TV & 386769 & 12.89\% & \$26.01 \\
3 & CDs \& Vinyl & 264310 & 8.81\% & \$18.10 \\
4 & Buy a Kindle & 208003 & 6.93\% & \$10.60 \\
5 & Toys \& Games & 142277 & 4.74\% & \$44.52 \\
6 & Video Games & 81003 & 2.70\% & \$48.13 \\
7 & Sports \& Outdoors & 76039 & 2.53\% & \$61.28 \\
8 & Home Improvement & 59606 & 1.99\% & \$65.00 \\
9 & All Electronics & 48746 & 1.62\% & \$126.75 \\
10 & Amazon Fashion & 48251 & 1.61\% & \$69.00 \\
\end{tabular}
\end{adjustbox}
\end{table}


\begin{table}[t]
\centering
\caption{Female User Preference}
\label{tb:female}
\begin{adjustbox}{max width=.5\textwidth}
\begin{tabular}{lllll}
Rank & Item Type          & Number of Items & Percentage of Items & Average Price(\$) \\ \hline
1 & Books & 493964 & 37.22\% & \$17.50 \\
2 & Buy a Kindle & 176831 & 13.32\% & \$8.00 \\
3 & Movies \& TV & 137732 & 10.38\% & \$23.30 \\
4 & Toys \& Games & 82399 & 6.21\% & \$33.71 \\
5 & CDs \& Vinyl & 68012 & 5.12\% & \$14.65 \\
6 & Amazon Fashion & 51788 & 3.90\% & \$48.08 \\
7 & Kitchen \& Dining & 49058 & 3.70\% & \$38.30 \\
8 & Home \& Kitchen & 42372 & 3.19\% & \$50.04 \\
9 & All Beauty & 25440 & 1.92\% & \$21.74 \\
10 & Arts, Crafts \& Sewing & 21475 & 1.62\% & \$22.51 \\
\end{tabular}
\end{adjustbox}
\end{table}

\subsubsection{Different Region}

After presenting strong variances in gender preference with quantitative analysis. We now show geographical difference among 3 regions as mentioned -- east coast, west coast, and the US. Note that only \%57 users have location information in their list descriptions. Following analysis include only the users that input their locations. Due to space limitation we do not show all popular products. Instead we present some of the interesting observations. 

The total amount of items from east coast, west coast, and mid of US are 641,252, 420,828, and 1,109,846. The average product price in east coast, west coast, and mid of US are \$35.80, \$35.83, and \$33.45. While users from east coast and west coast have almost the same product price, users in the middle of US expose highest price sensitivity -- they prefer to pay 6.6\% less than other 2 groups in products they desire. It also agrees with the common knowledge that coastal areas have higher income than mid area. However, generally the distributions of items are very similar in the 3 regions. Some noteworthy differences are (1) users from east coast has 2.66\% ``sports \& Outdoors" products. The percentage is 14.2\% and 33.0\% more than that of west coast and mid of the U.S. However, they are prone to pay 8.4\% and 7.9\% less in these products. (2) West coastal users are willing to pay \$70.88 in ``Home Improvement" products, which are 12.9\% and 22.8\% more than east and mid area. The percentage of ``Home Improvement" is also higher (1.92\% compared to 1.49\% in mid area and 1.72\% in east coast). 

To conclude, we show that different genders expose vastly different online shopping preference. We also show that people from 3 different regions have similar shopping pattern. Although in most cases it is true, we still point out noticeable diversity among users in these 3 regions. 

\subsection{Time Factor}
After analyzing product categories and price, we study another important factor that describes user behaviors -- time factor. Learned from 5,710,674 items, we found users start to add items to wishlists since 1999. There are only 2,056 items added in 1999. However, in each year the items added increase 85.6\% in average. This rapid increase is reasonable since electronic commercials are gaining popularity. Next we explore shopping trend during weekdays and weekends, as well as normal days and holidays. By investigating the added date of items we are able to learn the days that are most appealing to shoppers.

\subsubsection{Weekdays and Weekends}
Intuitively users may browse and shop online more on weekends since they have more free time. Surprisingly our analysis indicates that this intuition is in fact premature. Generally 3,905,728 and 1,523,764 items are added in 4,174 weekdays and 1,670 weekends from 1999 to 2014. Therefore averagely 935.7 items are added on a weekday and 912.4 items are added on a weekend, which implies weekend has 2.55\% less added items. We further present our result in terms of each year to show the general trend in Figure~\ref{fig:week}. As the figure shows, the gap between items in weekdays and weekends is large in earlier years but keeps shrinking. Items added in weekends start to exceed weekdays after 2012. However, there is no evidence showing that weekends would keep growing faster than weekdays. Therefore we conclude that currently people browse and shop online almost equally during weekdays and weekends. A reasonable explanation for this equality is that online shopping consume little effort from users -- all the they need to do is to sit in front of a computer at any time -- so they are not motivated to devote large chunks of free time in weekends to online shopping.

\begin{figure}[t]
\centering
  \caption{Items Added in Weekdays and Weekends}{}
  \label{fig:week}
  \centering
    \includegraphics[width=0.45\textwidth]{weekday.png}
\end{figure}

\subsubsection{Normal days and Holidays}
After studying the shopping favor in weekdays and weekends, similarly we measure holidays and normal days. As there are too many unofficial and regional holidays to study, our study focuses on nation-wide federal holidays. There are totally 10 qualified holidays~\footnote{http://www.archives.gov/news/federal-holidays.html}, which are listed in Table~\ref{tb:holiday}.

\begin{table}[t]
\centering
\caption{U.S Federal Holidays}
\label{tb:holiday}
\begin{adjustbox}{max width=.45\textwidth}
\begin{tabular}{ll}
New Year's Day & January 1  \\
Birthday of Martin Luther King, Jr. & third Monday in January  \\
Washington's Birthday & third Monday of February  \\
Memorial Day & last Monday of May \\
Independence Day & July 4  \\
Labor Day & first Monday of September  \\
Columbus Day & second Monday in October  \\
Veterans Day & November 11  \\
Thanksgiving Day & fourth Thursday in November  \\
Christmas Day & December 25  \\
\end{tabular}
\end{adjustbox}
\end{table}

When shopping in holidays, users may not always buy products on the exact date. For example, People usually prepare gifts before Thanksgiving. Therefore, we consider the nearest consecutive 5 days in our holiday shopping period (previous 2 days and next 2 days). Any day in this period will be counted as shopping in the holiday. For example, We include December 23 to December 27 to calculate the shopping behavior in Christmas day. Note that as a side effect of our methodology, the last 2 days of the previous year are also included when analyzing New Year's day. However, it does not affect our results since we are only interested in the average items in normal days and holidays.

First, we show the average number of items in normal days and holidays in each year as well as the increase rate in Figure~\ref{fig:holiday}. Interestingly we found that there is very limited increase from normal days to holidays in average. The average increase is 10.14\%. However, as the data of year 1999 is not sufficient to be representative (Only 2,056 items in 1999), we exclude this year in calculation. Therefore the average increase becomes 5.8\%. We then conclude that generally people do shop online more during holidays. However, the increase in holidays is merely 5.8\%. 

Our findings are against common impression that people do shop a lot in holidays, especially Thanksgiving and Christmas. We realized that it is rough to group all national holidays together to compare to normal days. Instead we could treat holiday individually since they are also different from each other. Therefore we compute the average number of items added in each of the holiday. Similar to previous general study, we count the nearest consecutive 5 days of the holiday date as the holiday period. To show both the general case and the current trend, we use the data from 1999 to 2014 and the year 2014 solely. We do not use data from year 2015 because during the time of this study, year 2015 has not been finished. Note that when computing the average item number added in normal days, we leave out all the holiday periods instead of the one that is being analyzed. We show the result in Figure~\ref{fig:diffholiday}. Clearly we can see that holidays are very different from each other. Based on the figure we also made following interesting observations.

\begin{figure}[t]
\centering
  \caption{Items Added in Normal Days and Holidays}{}
  \label{fig:holiday}
  \centering
    \includegraphics[width=0.45\textwidth]{holiday.png}
\end{figure}

\begin{enumerate}[leftmargin=*]
\item The sub-figures for year 2014 and average case match well, which indicates that people do not tend to change their shopping behaviors in the long run. 
\item New Year's day, Veterans day, Thanksgiving, and Christmas day are 4 holidays that have quite obvious shopping increase. The result indicates that people tend to buy more stuff during the 4 holidays. The increase rate is 4.1\%, 25.2\%,72.8\%, and 30.2\% accordingly for all 15 years we analyzed. 
\item Among the 4 holidays, Thanksgiving day is the most shopping appealing holiday, which is 72.8\% higher than normal days. Christmas ranks second with 30.2\% increase. We believe the result is reasonable because people always get considerable discount during thanksgiving. Besides, Thanksgiving and Christmas usually involve lots of presents.
\item Some of the holidays are close to normal days in terms of number of added items in user wishlists (For example, Columbus Day). Furthermore, some of the holidays make the people less willing to add items in their wish lists. 4 holidays -- Birthday of Martin Luther King. Jr, Washington's Birthday, Memorial Day, Independence Day, and Labor day --  have clearly less items added to user Wish Lists. The drop percentage is 10.7\%, 11.8\%, 9.0\%, 11.3\%, and 4.6\% accordingly. One possible reason is that People are more likely to be engaged in other activities other than shopping during these holidays since 3 out of the 4 holidays are memorial-type days.

\end{enumerate}

\begin{figure}[t]
\centering
  \caption{Items Added in Different Holidays}{}
  \label{fig:diffholiday}
  \centering
    \includegraphics[width=0.45\textwidth]{holidayfig.png}
\end{figure}

\subsection{Result Implications}
Up to this point, we have analyzed 3 aspects regarding to user online shopping behaviors in Amazon. They are product categories and price users prefer in general and on a gender basis. And time factor that affects user shopping behaviors Since Amazon is the largest electronic commercial in the world that offers almost all kinds of products, we believe our analysis is also able to shed light on the general case in e-commerce industry. 

Our insights on user shopping patterns can be used both in general education purpose and commercial related purpose. Understanding user behavior can be an important factor in e-commerce ecosystem. Equipped with the knowledge, companies are able to better investigate the potential of products, refine marketing strategies, and do timely promotions. Besides, our result show user preference from different user groups, it can also help develop more accurate targeted advertising.

