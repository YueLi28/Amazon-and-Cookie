\section{Discussion}
\subsection{Data Limitation}
There is certain limitation in the data of this study. One problem is that the data may be biased. User wishlists are not always public -- users have the ability to change the accessibility of their wish lists (although the default is public). Therefore privacy-aware users may choose to publicize some of their wish lists while keep other wish lists from strangers. Besides, users may choose not to share certain items in their wish lists. For example, privacy-sensitive items such as pregnancy test, firearm-related products, and medicine \& drugs. In our work we can only retrieve the product in public wish lists. The data may not be 100\% representative of one users' shopping behavior.

\subsection{Personal Information Identification}
We have done a pilot study on predicting user gender from their wishlists. We now discuss the potential to use wishlists to identify other types of personal information. To conduct such experiments, first we may need to identify the ground truth. Other than the several available information, we can extract useful information in the list-descriptions. However, it involves accurate human language processing and may require much more effort. Some other way to obtain ground truth is to search online social network, which is shown to be fairly easy~\cite{krishnamurthy2009leakage}. In order to maintain high accuracy, finer-granularity wishlist information may be retrieved. For example, subcategories, ranking of items, timing factor could be effective features to create better machine learning models. We believe identifying more personal information using wishlists is very promising, which will be left as a future work. 

